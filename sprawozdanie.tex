\documentclass[a4paper,12pt,openany]{report}

\usepackage[a4paper, total={18cm, 27cm}]{geometry}
\usepackage{layout}
\usepackage[utf8]{inputenc}
\usepackage{polski}
\usepackage{graphicx}
\usepackage{enumitem}
\usepackage{amsmath}
\usepackage{index}
\usepackage{subcaption}
\usepackage{hyperref}
\usepackage{titlesec}
\hypersetup{
    colorlinks=true,
    linkcolor=blue,
    filecolor=magenta,      
    urlcolor=cyan,
    pdftitle={Overleaf Example},
    pdfpagemode=FullScreen,
    }
\titleformat{\chapter}[display]{\normalfont\bfseries}{}{0pt}{\Large}
\tolerance=1
\emergencystretch=\maxdimen
\hyphenpenalty=10000
\hbadness=10000

  
\begin{document}

%%%%%%%%%%%%%%%%%%%%%%%%%%%%%%%%%%%%%%%%%%%%%%%%%%%%%%%%%%%%%%%%%%%%%%%%%%%%%%%%%%%%%%%%%%%%%%%%%%%%%%%%%%%%%%%%%%%%
%%%%%%%%%%%%%%%%%%%%%%%%%%%%%%%%%%%%%%%%%%%%%%%%%%%%%%%%%%%%%%%%%%%%%%%%%%%%%%%%%%%%%%%%%%%%%%%%%%%%%%%%%%%%%%%%%%%%
\title{\Large{\textbf{Wariacje filtrów uśredniających i wyostrzających w analizie obrazów.}}}
\author{Filip Matejko 249965}
\date{13.01.2022}
\maketitle

\tableofcontents

%%%%%%%%%%%%%%%%%%%%%%%%%%%%%%%%%%%%%%%%%%%%%%%%%%%%%%%%%%%%%%%%%%%%%%%%%%%%%%%%%%%%%%%%%%%%%%%%%%%%%%%%%%%%%%%%%%%%
%%%%%%%%%%%%%%%%%%%%%%%%%%%%%%%%%%%%%%%%%%%%%%%%%%%%%%%%%%%%%%%%%%%%%%%%%%%%%%%%%%%%%%%%%%%%%%%%%%%%%%%%%%%%%%%%%%%%
\chapter{Wstęp}

\paragraph{\indent Coraz częściej do analizy obrazów wykorzystuje się komputery. Jednym ze sposobów poprawy skuteczności pracy systemów do analizy obrazów jest stosowanie filtrów. Filtry dobiera się w odpowiednio w zależności od tego co chce się osiągnąć.}

\paragraph{\indent W tej pracy opisuję działanie 2 wybranych filtrów - uśredniający (wygładzający) i wyostrzający. Analizie poddam również wpływ wielkości i kształtu filtrów na efekt końcowy.}

\paragraph{\indent W celu pokazania działania filtrów wykorzystuję je w splocie ze zdjęciem Lenny. Zdjęcie te jest wykorzystywane do testowania algorytmów do analizy obrazów ze względu na dużą ilość elementów sprawiających trudności - gładkie przejścia, ostre krawędzie itp \href{https://en.wikipedia.org/wiki/Lenna}{[1]}. W celu ułatwienia wykonania splotu i analizy filtrów wykorzystuję zdjęcie w wersji czarno białej. W takiej sytuacji filtr ma postać macierzy dwuwymiarowej. W splocie zdjęcia kolorowego należałoby wykorzystać macierz trójwymiarową (ze względu na 3 kanały - RGB).}

\begin{center}
\includegraphics[width=10cm]{resources/Lena.jpg}\\
\tiny{Lenna}
\end{center}

\chapter{Filtry}

\section{Filtr uśredniający}
\text{Przykład filtru uśredniającego:}
\\
\linebreak
$\begin{pmatrix}
0.25 & 0.25\\
0.25 & 0.25
\end{pmatrix}$
\\
\linebreak
\text{Wyznaczanie wartości w macierzy:}
\\
\linebreak
\begin{gather*}
W - \text{liczba wierszy} \\
K - \text{liczba kolumn} \\
n[w, k] - \text{wartość macierzy w wierszu 'w' i kolumnie 'k'} \\
n[w, k] = \frac{1}{w*k}\\
\end{gather*}

\section{Filtr wyostrzający}
\text{Przykład filtru uśredniającego:}
\\
\linebreak
$\begin{pmatrix}
0 & -1 & 0 \\
-1 & 5 & -1 \\
0 & -1 & 0 \\
\end{pmatrix}$
\\
\linebreak
\text{liczba wierszy i kolumn macierzy musi być nieparzysta.}
\\
\linebreak
\text{Wyznaczanie wartości w macierzy:}
\\
\linebreak
\begin{gather*}
W - \text{nieparzysta liczba wierszy} \\
K - \text{nieparzysta liczba kolumn} \\
n[w, k] - \text{wartość macierzy w wierszu 'w' i kolumnie 'k'} \\
\\
\text{jeśli w = W//2 oraz k = K//2 to:}\\
n[w, k] = W + K - 1\\
\\
\text{jeśli w = W//2 albo k = K//2 to:}\\
n[w, k] = -1\\
\\
text{pozostałe przypadki to:}\\
n[w, k] = 0\\
\end{gather*}


%%%%%%%%%%%%%%%%%%%%%%%%%%%%%%%%%%%%%%%%%%%%%%%%%%%%%%%%%%%%%%%%%%%%%%%%%%%%%%%%%%%%%%%%%%%%%%%%%%%%%%%%%%%%%%%%%%%%
%%%%%%%%%%%%%%%%%%%%%%%%%%%%%%%%%%%%%%%%%%%%%%%%%%%%%%%%%%%%%%%%%%%%%%%%%%%%%%%%%%%%%%%%%%%%%%%%%%%%%%%%%%%%%%%%%%%%
\chapter{Działanie filtrów}

\paragraph{\indent Działanie filtru uśredniającego oraz wyostrzającego są analogiczne. Okno wielkości filtra przesuwa się po obrazie, jego zawartość jest wymnażana przez macierz filtra, a następnie sumowana. Przykład splotu z filtrem uśredniającym:}

\begin{center}
\includegraphics[width=13cm]{resources/blur1.png}
\linebreak
\tiny{Krok 1.}
\end{center}

\begin{center}
\includegraphics[width=13cm]{resources/blur2.png}
\linebreak
\tiny{Krok 2.}
\end{center}

\begin{center}
\includegraphics[width=13cm]{resources/blur3.png}
\linebreak
\tiny{Krok 3.}
\end{center}


\pagebreak
\paragraph{\indent Wartości na brzegach są otrzymywane przez rozszerzenie obrazu. Sposób wykorzystany w moim rozwiązaniu polega na odbiciu wartości przy krawędzi obrazu. Przykład odbicia względem prawej krawędzi obrazu:}

\begin{center}
\includegraphics[width=10cm]{resources/reflect.png}
\linebreak
\tiny{Niebieskie pola wskazują obraz, fioletowe - odbicie.}
\end{center}

%%%%%%%%%%%%%%%%%%%%%%%%%%%%%%%%%%%%%%%%%%%%%%%%%%%%%%%%%%%%%%%%%%%%%%%%%%%%%%%%%%%%%%%%%%%%%%%%%%%%%%%%%%%%%%%%%%%%
%%%%%%%%%%%%%%%%%%%%%%%%%%%%%%%%%%%%%%%%%%%%%%%%%%%%%%%%%%%%%%%%%%%%%%%%%%%%%%%%%%%%%%%%%%%%%%%%%%%%%%%%%%%%%%%%%%%%
\chapter{Przykładowe wyniki filtracji}

\begin{center}
\includegraphics[width=15cm]{resources/modified/lena_gray.jpg}
\linebreak
\tiny{Czarno biała wersja zdjęcia źródłowego poddawana filtracji.}
\end{center}

%%%%%%%%%%%%%%%%%%%%%%%%%%%%%%%%%%%%%%%%%%%%%%%%%%%%%%%%%%%%%%%%%%%%%%%%%%%%%%%%%%%%%%%%%%%%%%%%%%%%%%%%%%%%%%%%%%%%%
\pagebreak
\section{Filtr uśredniający}

\begin{center}
\includegraphics[width=7cm]{resources/modified/lena/lena_blur_3x3.jpg}
\linebreak
\tiny{Wynik działania filtru 3x3}
\end{center}

\begin{center}
\includegraphics[width=7cm]{resources/modified/lena/lena_blur_20x20.jpg}
\linebreak
\tiny{Wynik działania filtru 20x20}
\end{center}

\begin{center}
\includegraphics[width=7cm]{resources/modified/lena/lena_blur_40x40.jpg}
\linebreak
\tiny{Wynik działania filtru 40x40}
\end{center}

\pagebreak
\paragraph{\indent Jak można zauważyć na powyższych obrazach, rozmycie w stosunku do obrazu źródłowego jest tym większe, im większe wymiary ma wykorzystana macierz. W filtrach o większej powierzchni na jeden piksel obrazu wyjściowego przypada średnia wartość większej ilości pikseli obrazu źródłowego (np dla filtra 40x40 będzie to 1600 pikseli). Skutkuje to zwiększonym wygładzeniem sygnału, co w tym przypadku oznacza rozmycie obrazu. Poniżej przykład na prostszym obrazie:}

\begin{center}
\text{wyniki działania filtrów 40x40 oraz 20x20:}\\
\includegraphics[width=7cm]{resources/modified/sample/sample_blur_40x40.jpg}
\includegraphics[width=7cm]{resources/modified/sample/sample_blur_20x20.jpg}
\end{center}

\pagebreak
\subsection{Dodatkowe filtry}

\begin{center}
\text{wyniki działania filtrów 3x40 oraz 40x3:}\\
\includegraphics[width=7cm]{resources/modified/lena/lena_blur_3x40.jpg}
\includegraphics[width=7cm]{resources/modified/lena/lena_blur_40x3.jpg}
\\
\text{wyniki działania filtrów 20x3 oraz 3x20:}\\
\includegraphics[width=7cm]{resources/modified/lena/lena_blur_20x3.jpg}
\includegraphics[width=7cm]{resources/modified/lena/lena_blur_3x20.jpg}
\\
\text{wyniki działania filtrów 40x3 oraz 40x5:}\\
\includegraphics[width=7cm]{resources/modified/lena/lena_blur_40x3.jpg}
\includegraphics[width=7cm]{resources/modified/lena/lena_blur_40x5.jpg}
\end{center}

\pagebreak
\paragraph{\indent Dodatkowym zjawiskiem powstającym tylko dla niektórych filtrów (nie kwadratowych) jest zwiększone rozmycie wzdłuż jednej z osi. Jeśli filtr ma szerokość większą niż wysokość obraz zostanie bardziej rozmyty w osi poziomek. Jeśli filtr ma wysokość większą niż szerokość obraz zostanie bardziej rozmyty w osi pionowej. Im większy jest stosunek wysokości do szerokości tym efekt ten będzie bardziej wyraźny. Wynika to z faktu, że do kolejnych pikseli obrazu wyjściowego trafia więcej danych z jego 'sąsiadów' wzdłuż danej osi. Poniżej przykład na prostszym obrazie:}

\begin{center}
\text{wyniki działania filtrów 40x1 oraz 1x40:}\\
\includegraphics[width=7cm]{resources/modified/sample/sample_blur_40x1.jpg}
\includegraphics[width=7cm]{resources/modified/sample/sample_blur_1x40.jpg}
\\
\text{wyniki działania filtrów 40x5 oraz 5x40:}\\
\includegraphics[width=7cm]{resources/modified/sample/sample_blur_40x5.jpg}
\includegraphics[width=7cm]{resources/modified/sample/sample_blur_5x40.jpg}
\end{center}

%%%%%%%%%%%%%%%%%%%%%%%%%%%%%%%%%%%%%%%%%%%%%%%%%%%%%%%%%%%%%%%%%%%%%%%%%%%%%%%%%%%%%%%%%%%%%%%%%%%%%%%%%%%%%%%%%%%%%%%
\pagebreak
\section{Filtr wyostrzający}

\begin{center}
\includegraphics[width=7cm]{resources/modified/lena/lena_sharpen_3x3.jpg}
\linebreak
\tiny{Wynik działania filtru 3x3}
\end{center}

\begin{center}
\includegraphics[width=7cm]{resources/modified/lena/lena_sharpen_5x5.jpg}
\linebreak
\tiny{Wynik działania filtru 5x5}
\end{center}

\begin{center}
\includegraphics[width=7cm]{resources/modified/lena/lena_sharpen_11x11.jpg}
\linebreak
\tiny{Wynik działania filtru 11x11}
\end{center}

\pagebreak
\paragraph{\indent Działanie filtrów wyostrzających powoduje powstanie dodatkowych punktów kontrastu tym silniejszych im większa była różnica wartości między pikselami na osi oraz pikselem środkowym. W przypadku filtrów małych filtrów (np 3x3 i 5x5) podkreślone zostają mniej widoczne krawędzie na obrazie (np w piórach w kapeluszu). Zastosowanie większych filtrów powoduje powstanie dużego szumu na całym obrazie oraz znacznych zniekształceń w miejscach wyraźnym krawędzi. Poniżej przykłady na prostszym obrazie :}

\begin{center}
\text{wyniki działania filtrów 3x3 oraz 7x7:}\\
\includegraphics[width=7cm]{resources/modified/sample/sample_sharpen_3x3.jpg}
\includegraphics[width=7cm]{resources/modified/sample/sample_sharpen_7x7.jpg}
\\
\text{wyniki działania filtrów 11x11 oraz 21x21:}\\
\includegraphics[width=7cm]{resources/modified/sample/sample_sharpen_11x11.jpg}
\includegraphics[width=7cm]{resources/modified/sample/sample_sharpen_21x21.jpg}
\end{center}

\pagebreak
\subsection{Dodatkowe filtry}

\begin{center}
\text{wyniki działania filtrów 3x11 oraz 11x3:}\\
\includegraphics[width=7cm]{resources/modified/lena/lena_sharpen_3x11.jpg}
\includegraphics[width=7cm]{resources/modified/lena/lena_sharpen_11x3.jpg}
\\
\text{wyniki działania filtrów 7x7 oraz 9x9:}\\
\includegraphics[width=7cm]{resources/modified/lena/lena_sharpen_7x7.jpg}
\includegraphics[width=7cm]{resources/modified/lena/lena_sharpen_9x9.jpg}
\\
\text{wyniki działania filtrów 11x11 oraz 21x21:}\\
\includegraphics[width=7cm]{resources/modified/lena/lena_sharpen_11x11.jpg}
\includegraphics[width=7cm]{resources/modified/lena/lena_sharpen_21x21.jpg}
\end{center}

\pagebreak
\paragraph{\indent W przeciwieństwie do filtrów uśredniających, filtry wyostrzające nie zmieniają tak mocno skutków działania w zależności od stosunku szerokości do wysokości. Poniżej przykłady na prostszym obrazie :}

\begin{center}
\text{wyniki działania filtrów 40x1 oraz 1x40:}\\
\includegraphics[width=7cm]{resources/modified/sample/sample_sharpen_21x3.jpg}
\includegraphics[width=7cm]{resources/modified/sample/sample_sharpen_3x21.jpg}
\\
\text{wyniki działania filtrów 40x5 oraz 5x40:}\\
\includegraphics[width=7cm]{resources/modified/sample/sample_sharpen_5x11.jpg}
\includegraphics[width=7cm]{resources/modified/sample/sample_sharpen_11x5.jpg}
\end{center}

%%%%%%%%%%%%%%%%%%%%%%%%%%%%%%%%%%%%%%%%%%%%%%%%%%%%%%%%%%%%%%%%%%%%%%%%%%%%%%%%%%%%%%%%%%%%%%%%%%%%%%%%%%%%%%%%%%%%
%%%%%%%%%%%%%%%%%%%%%%%%%%%%%%%%%%%%%%%%%%%%%%%%%%%%%%%%%%%%%%%%%%%%%%%%%%%%%%%%%%%%%%%%%%%%%%%%%%%%%%%%%%%%%%%%%%%%
\chapter{Wnioski}

\paragraph{\indent Wykorzystując odpowiednie filtry należy pamiętać o tym jak zależą jego efekty w stosunku do jego wielkości. Filtr wykorzystany nieoptymalnie może dać odwrotne efekty od oczekiwanych. W celu lepszego poznania charakterystyki filtru można wykonać kilka testów na rożnych obrazach w różnych wariacjach filtru. Dodatkowo warto wiedzieć jak działają filtry górnoprzepustowe (np filtr wyostrzający oraz filtry dolno przepustowe (np filtr wygładzający).}


%%%%%%%%%%%%%%%%%%%%%%%%%%%%%%%%%%%%%%%%%%%%%%%%%%%%%%%%%%%%%%%%%%%%%%%%%%%%%%%%%%%%%%%%%%%%%%%%%%%%%%%%%%%%%%%%%%%%%%%%%%%
\chapter{Dodatkowe}
\paragraph{Do wykonania zamieszczonych obrazów wykorzystałem napisane przeze mnie skrypty. Wszystkie skrypty oraz dodatkowe przefiltrowane obrazy są zamieszczone w \href{https://github.com/FilipM13/CPS}{moim repozytorium na githubie}.}

\end{document}